\documentclass[10pt]{article}

\usepackage[margin=2.0cm,paperwidth=180mm,paperheight=1950mm]{geometry}
\usepackage{amssymb}
\usepackage{amsmath}
\usepackage{amstext}
\usepackage{amsthm}
\usepackage{esint} % more integral signs 
\usepackage{float}
\usepackage{graphicx}
\usepackage{setspace}
\usepackage{url}
\usepackage[affil-it]{authblk}

\newtheorem{theorem}{Theorem}
\newtheorem{definition}{Definition}
\setstretch{1.1}

\begin{document}
\title{An Exploration of a Bijective Mapping in the Integers}
\author{John Rugis\thanks{Maraetai, New Zealand. \emph{jrugis@gmail.com}}}
\affil{(A revision of the original April 5, 2007 publication.)}
\date{December 18, 2016}
\maketitle
\thispagestyle{empty}

\begin{abstract}
In this article we begin by considering the possible existence of two bijective integer valued functions where the sum of the functions is also bijective.
An example of two such functions is produced.
Some interesting properties of these functions, as well as the algorithm that is used to produce them, are both explored. 
\end{abstract}

\section{Preliminaries}
Given two functions $f:\mathbb Z\!\rightarrow\!\mathbb Z$ and $g:\mathbb Z\!\rightarrow\!\mathbb Z$,
define a new function $(f\!+\!g):\mathbb Z\!\rightarrow\!\mathbb Z$ by $(f\!+\!g)(n)=f(n)+g(n)$. 
Is it possible that $f$ and $g$ as well as $f\!+\!g$ could all be bijective?
We will prove that it is possible by producing an example.\\

We begin by sequentially selecting and plotting points $(f(n),g(n))$
on a graphic grid that represents $\mathbb Z^2$.
We will work our way through all of $\mathbb Z$ by considering $n$'s in the order
$0,+1,-1,+2,-2,+3,-3,\dots$
and impose three sufficient conditions.\\

\noindent\textbf{Selection Conditions:}
\begin{enumerate}
\item
We will select points $(f(n),g(n))\!\in\! \mathbb Z^2$ such that for all $n \!\in\! \mathbb Z$, $f(n)+g(n)=n$.
This will ensure that $(f\!+\!g)(n)$ is bijective.
\item
If we are careful in our selection of points $(f(n),g(n))$ making sure that
$f(n)$ never maps to the same value twice and
$g(n)$ never  maps to the same value twice,
then both $f(n)$ and $g(n)$ will be injective.
\item
And finally, if we are also careful in our selection of points $(f(n),g(n))$ making sure that $f(n)$ and $g(n)$ map to all elements of $\mathbb Z$,
then $f(n)$ and $g(n)$ will be surjective.
\end{enumerate}
\vspace{0.4cm}
In the next section we produce an example that meets these conditions.\\

\section{A Mapping Algorithm}
\begin{figure}[H]
\begin{center}
\includegraphics[width=0.9\textwidth]{Z2_00.pdf}\\
\end{center}
\caption{Beginning at the point plot of $(f(n),g(n))$ in $Z^2$ on the left. The first dashed \emph{keep-off} lines are shown on the right.}
\label{fig:00}
\end{figure}

Starting with $n=0$, we note that, to meet condition 1, the point $(f(0),g(0))$ will need to be on the line labeled $n=0$ in Figure \ref{fig:00}. 
Let's (somewhat arbitrarily) start by selecting the point $(0,0)$.
Next we draw two dashed \emph{keep-off} lines $f(n)=0$ and $g(n)=0$.
If we make sure that no other points that are selected lie on a keep-off line, then we will have met condition 2.\\

\begin{figure}[H]
\begin{center}
\includegraphics[width=0.9\textwidth]{Z2_01.pdf}\\
\end{center}
\caption{Evolving point selection.}
\label{fig:01}
\end{figure}

\begin{figure}[H]
\begin{center}
\includegraphics[width=0.45\textwidth]{Z2_02.pdf}\\
\end{center}
\caption{Plotted points up to $n=\pm6$.}
\label{fig:02}
\end{figure}

As we work our way through values of $n$, we note that the points we select need to lie on their respective diagonal line as shown in the evolving plot Figure \ref{fig:01}.
In a systematic effort to meet condition 3, we will select points that are as close to $(0,0)$ as possible.
So, for $n=1$ we select the point $(2,-1)$ and for $n=-1$ we select the point $(-2,1)$ as shown in Figure \ref{fig:01}.
And then, following the same rule, for $n=2$ we select the point $(-1,3)$ and for $n=-2$ we select the
point $(1,-3)$.
As we continue systematically plotting points, we make the side observation that the plotted points appear to closely fit two lines passing through zero (see Figure \ref{fig:02}).\\

\begin{table}[H]
\begin{center}
\begin{tabular}{c|c|c|c|c}
\multicolumn{2}{c}{$n$ odd} &\multicolumn{1}{c}{} &\multicolumn{2}{c}{$n$ even}\\
$f(n)$ &$g(n)$ &$n$ &$f(n)$ &$g(n)$ \\
\hline
 & &0 &0 &0 \\ \hline
2 &-1 &1 & & \\ \hline
 & &2 &-1 &3 \\ \hline
5 &-2 &3 & & \\ \hline
 & &4 &-3 &7 \\ \hline
9 &-4 &5 & & \\ \hline
 & &6 &-4 &10 \\ \hline
12 &-5 &7 & & \\ \hline
 & &8 &-6 &14 \\ \hline
15 &-6 &9 & & \\ \hline
 & &10 &-7 &17 \\ \hline
19 &-8 &11 & & \\ \hline
 & &12 &-9 &21 \\ \hline
22 &-9 &13 & & \\ \hline
 & &14 &-10 &24 \\ \hline
26 &-11 &15 & & \\ \hline
 & &16 &-11 &27 \\ \hline
29 &-12 &17 & & \\
\end{tabular}
\end{center}
\caption{Tabulated values for $0\le n\le 17$.}
\label{tab:00}
\end{table}

In an attempt to discover an algebraic (as opposed to geometric) definition for our sequence of $f(n)$ and $g(n)$ values, we continue with a list of values as shown in Table 1.
Note that because of the symmetries $f(-n) = -f(n)$ and $g(-n) = -g(n)$, we can, without loss of generality, use a table that gives values only for $n\ge0$ as shown in Table \ref{tab:00}. 
The table has been split into columns for odd and even values of $n$.
Note that when $n$ is odd, $f(n)$ is positive and $g(n)$ is negative.
For all even values of $n$ greater than zero, $f(n)$ is negative and $g(n)$ is positive.\\

Observation of the evolving pattern leads to the following algorithm for specifying the sequence of values for $f(n)$ and $g(n)$:\\

\noindent\textbf{Mapping Algorithm 1:} (A strongly recursive definition.)
\begin{enumerate}
\item
Initialise $n$ equal to zero.
\item
If $n$ is even, set $f(n)$ equal to $-1$ times the smallest non-negative integer
value not in the set of the absolute values of already used $f(n)$'s. 
Then assign $g(n)$ such that $f(n) + g(n) = n$.
\item
If $n$ is odd, set $g(n)$ equal to $-1$ times the smallest non-negative integer
value not in the set of the absolute values of already used $g(n)$'s.
Then assign $f(n)$ such that $f(n) + g(n) = n$.
\item
Increment $n$ by one and loop to step 2.
\end{enumerate}
\vspace{0.4cm}

Of course, by the previously mentioned symmetry conditions, to complete the mapping, for all negative $n$, $f(n) = -f(-n)$ and $g(n) = -g(-n)$.\\

The algorithm guarantees that both $f(n)$ and $g(n)$ are surjective (selection condition 3) by virtue of the criteria ``smallest non-negative integer value not in the set".
All values of $f(n)$ and $g(n)$ will eventually get filled-in.\\

To show that $f(n)$ and $g(n)$ are both injective (selection condition 2), we first note that and no duplicate values will be assigned by the ``smallest non-negative integer value not in the set" criteria.
Thus, when $n$ is odd, we need to show that, in algorithm step 3, $f(n)$ is always assigned a larger target value than the previous odd $n$ and also that a smaller value hole is left unused for algorithm step 2 to subsequently fill-in.\\

To do this it will suffice to show that for all odd $n$, $f(n + 2) > f(n) + 2$.\\

\begin{proof}
\hfill
\begin{enumerate}
\item Assume $n$ is odd.
\item By Mapping Algorithm 1, step 3, $f(n) = n-g(n)$.
\item Adding 2 to $n$, $f(n+2) = (n+2)-g(n+2)$.
\item By Mapping Algorithm 1, step 3, $g(n+2) < g(n)$.
\item Therefore, $-g(n+2) > -g(n)$.
\item Adding $(n + 2)$, $(n+2)-g(n+2) > (n+2)-g(n)$.
\item Regrouping, $(n + 2)-g(n+2) > (n-g(n))+2$.
\item Substituting from proof steps (2) and (3), $f(n+2) > f(n)+2$.
\end{enumerate}
\end{proof}

When $n$ is even, a similar argument can be applied to $g(n)$.
Thus, we have an algorithm that meets the selection conditions, thereby generating an instance of the bijective mapping that we set out to find.\\

\section{Further Exploration}
The fact that, as previously observed, the plotted points in Figure \ref{fig:02} appear to closely fit two lines, leads us to consider the possibility of an equivalent non-recursive definition for $f(n)$ and $g(n)$.
With the assistance of a computer, we calculated the values of $f(n)$ and $g(n)$ for $0\le n\le 10000001$.
Some of the results are shown in Table 2.\\

\begin{table}[H]
\begin{center}
\begin{tabular}{c|c|c|c}
$n$ &$f(n)$ &$g(n)$ &$g(n)/f(n)$\\ \hline
9999 &17069 &-7070 &-0.414201183432\\ \hline
10000 &-7071 &17071 &-2.41422712488\\ \hline
10001 &17073 &-7072 &-0.41422128507\\ \hline
99999 &170709 &-70710 &-0.414213661846\\ \hline
100000 &-70711 &170711 &-2.41420712478\\ \hline
100001 &170712 &-70711 &-0.414212240499\\ \hline
999999 &1707105 &-707106 &-0.414213536953\\ \hline
1000000 &-707107 &1707107 &-2.41421312475\\ \hline
1000001 &1707108 &-707107 &-0.414213394817\\ \hline
9999999 &17071066 &-7071067 &-0.414213558778\\ \hline
10000000 &-7071068 &17071068 &-2.41421352475\\ \hline
10000001 &17071070 &-7071069 &-0.414213578879\\ 
\end{tabular}
\end{center}
\caption{Some values generated by Mapping Algorithm 1.}
\label{tab:01}
\end{table}

The values of $f(n)$ and $g(n)$, especially when $n$ is a power of 10, look familiar!
Based on these observations, we conjecture that:\\

\noindent For even $n$,
\begin{align}
\lim_{n\to\infty}\frac{f(n)}{n} &= -\frac{1}{\sqrt2}\\
\lim_{n\to\infty}\frac{g(n)}{n} &= 1+\frac{1}{\sqrt2}\\
\lim_{n\to\infty}\frac{g(n)}{f(n)} &= -\left(1+\sqrt2\right)
\end{align}
and for odd $n$,
\begin{align}
\lim_{n\to\infty}\frac{f(n)}{n} &= 1+\frac{1}{\sqrt2}\\
\lim_{n\to\infty}\frac{g(n)}{n} &= -\frac{1}{\sqrt2}\\
\lim_{n\to\infty}\frac{g(n)}{f(n)} &= -\left(\frac{1}{1+\sqrt2}\right)
\end{align}
This suggests the following non-recursive equivalent to Mapping Algorithm 1:\\

\noindent\textbf{Mapping Algorithm 2:} (A non-recursive definition.)
\begin{enumerate}
\item
If $n$ is even,
\begin{align*}
f(n) &= -\lceil (\tfrac{1}{\sqrt{2}})n \rfloor\\
g(n) &= \lceil (1+\tfrac{1}{\sqrt{2}})n \rfloor
\end{align*}
\item
If $n$ is odd,
\begin{align*}
f(n) &= \lceil (1+\tfrac{1}{\sqrt{2}})n \rfloor\\
g(n) &= -\lceil (\tfrac{1}{\sqrt{2}})n \rfloor
\end{align*}
\end{enumerate}
\vspace{0.4cm}
Note that the operator pair $\lceil\;\rfloor$ means ``round to the nearest integer value".\\

\section{A Variation}
Let's leave the original problem as stated in the Introduction behind and consider a variation to Mapping Algorithm 1 in which $f(n)+g(n) = 2n$ rather than $n$.
Of course $f(n) + g(n)$ is no longer surjective and thus not bijective.\\

\noindent\textbf{Mapping Algorithm 3:} (A strongly recursive definition.)
\begin{enumerate}
\item
Initialise $n$ equal to zero.
\item
If $n$ is even, set $f(n)$ equal to $-1$ times the smallest non-negative integer
value not in the set of the absolute values of already used $f(n)$'s. 
Then assign $g(n)$ such that $f(n) + g(n) = 2n$.
\item
If $n$ is odd, set $g(n)$ equal to $-1$ times the smallest non-negative integer
value not in the set of the absolute values of already used $g(n)$'s.
Then assign $f(n)$ such that $f(n) + g(n) = 2n$.
\item
Increment $n$ by one and loop to step 2.
\end{enumerate}
\vspace{0.4cm}

\begin{table}[H]
\begin{center}
\begin{tabular}{c|c|c|c}
$n$ &$f(n)$ &$g(n)$ &$g(n)/f(n)$\\ \hline
9999 &26178 &-6180 &-0.23607609443\\ \hline
10000 &-6180 &26180 &-4.23624595469\\ \hline
10001 &26183 &-6181 &-0.236069205209\\ \hline
99999 &261801 &-61803 &-0.236068617003\\ \hline
100000 &-61803 &261803 &-4.236088863\\ \hline
100001 &261806 &-61804 &-0.236067928161\\ \hline
999999 &2618031 &-618033 &-0.236067869326\\ \hline
1000000 &-618034 &2618034 &-4.23606791859\\ \hline
1000001 &2618037 &-618035 &-0.236068092239\\ \hline
9999999 &26180337 &-6180339 &-0.236067969637\\ \hline
10000000 &-6180340 &26180340 &-4.23606791859\\ \hline
10000001 &26180343 &-6180341 &-0.236067991928\\ 
\end{tabular}
\end{center}
\caption{Some values generated by Mapping Algorithm 3.}
\label{tab:02}
\end{table}

Some calculated results are shown in Table \ref{tab:02}.
Again the values of $f(n)$ and $g(n)$ look familiar!
This time we see the golden ratio.
Based on this observation, we make the conjecture that, when using Mapping Algorithm 3:\\

Based on these observations, we conjecture that:\\

\noindent For even $n$,
\begin{align}
\lim_{n\to\infty}\frac{f(n)}{n} &= -\frac{1}{\phi}\\
\lim_{n\to\infty}\frac{g(n)}{n} &= 1+\phi\\
\lim_{n\to\infty}\frac{g(n)}{f(n)} &= -\phi^2-\phi
\end{align}
and for odd $n$,
\begin{align}
\lim_{n\to\infty}\frac{f(n)}{n} &= 1+\phi\\
\lim_{n\to\infty}\frac{g(n)}{n} &= -\frac{1}{\phi}\\
\lim_{n\to\infty}\frac{g(n)}{f(n)} &= \frac{1}{-\phi^2-\phi}
\end{align}
where $\phi = (1+\sqrt(5))/2$.\\

\section{A Generalisation}
Consider the following parameterised generalisation, for positive integers $k$, in
which $f(n) + g(n) = kn$.
Of course $f(n) + g(n)$ is only bijective when $k = 1$.\\

\noindent\textbf{Mapping Algorithm 4}: (A strongly recursive definition parameterised by $k\in \mathbb Z^+$.)
\begin{enumerate}
\item
Initialise $n$ equal to zero.
\item
If $n$ is even, set $f(n)$ equal to $-1$ times the smallest non-negative integer
value not in the set of the absolute values of already used $f(n)$'s. 
Then assign $g(n)$ such that $f(n) + g(n) = kn$.
\item
If $n$ is odd, set $g(n)$ equal to $-1$ times the smallest non-negative integer
value not in the set of the absolute values of already used $g(n)$'s.
Then assign $f(n)$ such that $f(n) + g(n) = kn$.
\item
Increment $n$ by one and loop to step 2.
\end{enumerate}
\vspace{0.4cm}

\begin{table}[H]
\begin{center}
\begin{tabular}{c|c|c|c}
$n$ &$f(n)$ &$g(n)$ &$g(n)/f(n)$\\ \hline
9999 &135179 &-5192 &-0.0384083326552\\ \hline
10000 &-5192 &135192 &-26.0385208012\\ \hline
10001 &135206 &-5193 &-0.038408058814\\ \hline
99999 &1351907 &-51920 &-0.0384050086286\\ \hline
100000 &-51920 &1351920 &-26.0385208012\\ \hline
100001 &1351934 &-51921 &-0.0384049813083\\ \hline
999999 &13519189 &-519202 &-0.0384048185139\\ \hline
1000000 &-519202 &13519202 &-26.0384243512\\ \hline
1000001 &13519216 &-519203 &-0.0384048157822\\ 
\end{tabular}
\end{center}
\caption{Some values generated when $k = 13$.}
\label{tab:03}
\end{table}

Using Mapping Algorithm 4, and supported by additional calculated observations (such as shown for $k = 13$ in Table \ref{tab:03}), we make the conjecture that:\\

\noindent For all $k\in \mathbb Z^+$, and $n$ even,
\begin{align}
\lim_{n\to\infty}\frac{f(n)}{n} &= \frac{-k}{(k-1)+\sqrt{k^2+1}} = \frac{1}{\psi_1}\\
\lim_{n\to\infty}\frac{g(n)}{n} &= k-\frac{1}{\psi_1}\\
\lim_{n\to\infty}\frac{g(n)}{f(n)} &= \frac{-1}{k+\sqrt{k^2+1}} = \frac{1}{\psi_2}
\end{align}
and for odd $n$,
\begin{align}
\lim_{n\to\infty}\frac{f(n)}{n} &= k-\frac{1}{\psi_1}\\
\lim_{n\to\infty}\frac{g(n)}{n} &= \frac{1}{\psi_1}\\
\lim_{n\to\infty}\frac{g(n)}{f(n)} &= \psi_2
\end{align}

We can use $\psi_1$ to show a connection between the two rather special numbers
$\sqrt{2}$ and the golden ratio:\\

\noindent When $k = 1$,
\begin{equation}
\sqrt{2} = -\psi_1
\end{equation}
and when $k = 2$,
\begin{equation}
\phi = -\psi_1
\end{equation}

Note that $\psi_2$ has the property that:
\begin{equation}
\psi_2 = -2k+\frac{1}{\psi_2}
\end{equation}
which is equivalent to saying that the fractional part of $\psi_2$ is equal to $1/\psi_2$.
This certainly appears to be supported by examining the $g(n)/f(n)$ column in each of the previous mapping tables.\\

There is also a connection between the generalised limits and solutions to certain quadratic equations.
Note that we have:
\begin{align}
\psi_1 &= \frac{-(k-1)-\sqrt{k^2+1}}{k} \\
\psi_2 &= -k-\sqrt{k^2+1}
\end{align}

These $\psi_1$ and $\psi_2$ are clearly solutions, respectively, to the following quadratic equations:
\begin{align}
kx^2-(2k-2)x-2 &= 0 \\
x^2+2fx-1 &= 0
\end{align}

\section{Conclusion}
We have produced a parameterised algorithm that generates functions which map integers to integers.
These functions give sequences, that, in the infinite limit, are conjectured to give some interesting results.
The results are related specifically to two special numbers: $\sqrt{2}$ and the golden ratio.
The results also generally relate a special case of a fractional part to its multiplicative inverse, as well as generally to the solution of certain quadratic equations.
\end{document}