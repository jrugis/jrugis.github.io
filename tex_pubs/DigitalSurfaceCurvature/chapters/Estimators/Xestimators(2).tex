The surface curvature of continuous smooth lines and surfaces is a well-defined property, 
however, when working with point set data, triangulated surfaces or 3D digital images, 
line and surface curvatures can only be estimated \cite{Klette_2004_DG}.\\

%===================================================
\section{Digitization}

Digitized 3D Objects.
The 2D surface of solid 3D objects.
Might need to segment solid data to extract surface data.\\

We can only \emph{estimate the properties} of the actual continuous object.
If we want to check the correctness of our estimates, additional information about the source object is required.\\

\begin{figure}[tbp]
\begin{center}
\epsfxsize 50mm
\epsfbox{chapters/estimators/shape1}
\hspace{20pt}
\epsfxsize 48mm
\epsfbox{chapters/estimators/shape3}
\end{center}
\vspace{-12pt}
\caption{Sphere, point cloud and voxel digitization.}
\label{fig-shape1-3}
\end{figure}

\begin{figure}[tbp]
\begin{center}
\epsfxsize 50mm
\epsfbox{chapters/estimators/shape2}
\hspace{10pt}
\epsfxsize 45mm
\epsfbox{chapters/estimators/shape5}
\hspace{10pt}
\epsfxsize 45mm
\epsfbox{chapters/estimators/shape4}
\end{center}
\vspace{-12pt}
\caption{Ellipsoid, point cloud and voxel digitization.}
\label{fig-shape2-4}
\end{figure}

%===================================================
\subsection{???}
Surface data is a set of points having position and value.
Position: integer or real valued  XYZ vector.
Value: single binary bit or integer vector (density, temperature, color, etc).
Also edge connectivity might be given: a mesh.\\

%===================================================
\subsection{Binary grid scan}
Gaussian digitization,
Grid aligned point scanning.
XYZ integer for location.
Binary ``present or not present" value.
Need to segment the data to identify surface points.

%===================================================
\subsection{Density scans}
XYZ integer grid.
Color coded real-valued density.

%===================================================
\subsection{Range finding scans}
Spherical coordinates.
Angle grid.
Real valued depth.
No color.

%===================================================
\subsection{Synthetic object data}
Information about the source object is no problem!

%===================================================
\subsection{Real-world scan data}
Noisy data.
Corrupted data.

%===================================================
\section{Planar Line Curvature}
\begin{figure}[tbp]
\begin{center}
\epsfxsize 50mm
\epsfbox{chapters/estimators/curve1}
\end{center}
\vspace{-12pt}
\caption{Planar line curvature estimation.}
\label{fig-curve1}
\end{figure}

With reference to Figure~\ref{fig-curve1}, the planar line curvature can be estimated as the incremental angular advance divided by the incremental change in length
\begin{equation}\kappa = \frac{\alpha}{(d_1+d_2)/2}
\end{equation}
where $d_1$ is the length of the line segment from $\mathbf{p}_1$ to $\mathbf{p}_2$and $d_2$ is the length of the line segment from $\mathbf{p}_2$ to $\mathbf{p}_3$. Consequently, the curvature can be calculated as
\begin{equation}
\kappa = \left(\frac{2}{||\mathbf{v}_1|| + ||\mathbf{v}_2||} \right) \:\cos^{-1} \!\! \left(\frac{\mathbf{v}_1 \cdot\mathbf{v}_2}{||\mathbf{v}_1|| \, ||\mathbf{v}_2||}\right)
\end{equation}
where $\mathbf{v}_1 = \mathbf{p}_2 - \mathbf{p}_1$ and$\mathbf{v}_2 = \mathbf{p}_3 - \mathbf{p}_2$.

%===================================================
\section{Gaussian Curvature}

%===================================================
\section{Mean Curvature}

%===================================================
\section{Orthogonal Cuts}
\label{sec-orthocut-est}
